% Pengaturan ukuran teks dan bentuk halaman dua sisi
\documentclass[10pt, twoside]{report}

% Pengaturan ukuran halaman dan margin
\usepackage[a5paper,top=25mm,left=25mm,right=20mm,bottom=25mm]{geometry}

% Pengaturan ukuran spasi
\usepackage[singlespacing]{setspace}

% Judul dokumen
\title{Buku Laporan Kerja Praktik ITC}
\author{Musk, Elon Reeve \and Kjellberg, Felix Arvid Ulf}

% Pengaturan format bahasa
\usepackage[indonesian]{babel}

% Pengaturan detail pada file PDF
\usepackage[pdfauthor={\@author},bookmarksnumbered,pdfborder={0 0 0}]{hyperref}

% Pengaturan jenis karakter
\usepackage[utf8]{inputenc}

% Package lainnya
\usepackage{etoolbox} % Mengubah fungsi default
\usepackage{enumitem} % Pembuatan list
\usepackage{lipsum} % Pembuatan template kalimat
\usepackage{graphicx} % Input gambar
\usepackage{longtable} % Pembuatan tabel
\usepackage[table,xcdraw]{xcolor} % Pewarnaan tabel
\usepackage{natbib} % Kutipan artikel
\usepackage{eso-pic} % Pembuatan background
\usepackage{changepage} % Pembuatan teks kolom
\usepackage{wrapfig} % Wrapping gambar

% Definisi untuk "Hati ini sengaja dikosongkan"
\def\kosong{
	\vspace*{\fill}
	\begin{center}\textit{Halaman ini sengaja dikosongkan}\end{center}
	\vfill
}
\patchcmd{\cleardoublepage}{\hbox{}}{\kosong}{}{}

% Pengaturan penomoran halaman
\usepackage{fancyhdr}
\fancyhf{}
\renewcommand{\headrulewidth}{0pt}
\pagestyle{fancy}
\fancyfoot[CE,CO]{\thepage}
\patchcmd{\chapter}{plain}{fancy}{}{}
\patchcmd{\chapter}{empty}{plain}{}{}

% Pengaturan format judul bab
\usepackage{titlesec}
\titleformat{\chapter}[display]{\bfseries\Large}{BAB \centering\Roman{chapter}}{0ex}{\vspace{0ex}\centering}[\vspace{2ex}]
\titleformat{\section}{\bfseries\large}{\MakeUppercase{\thesection}}{1ex}{}
\titleformat{\subsection}{\bfseries\large}{\MakeUppercase{\thesubsection}}{1ex}{}
\titleformat{\subsubsection}{\bfseries\large}{\MakeUppercase{\thesubsubsection}}{1ex}{}
\titlespacing*{\chapter}{0ex}{0ex}{0ex}
\titlespacing{\section}{0ex}{1ex}{1ex}
\titlespacing{\subsection}{0ex}{0.5ex}{0.5ex}
\titlespacing{\subsubsection}{0ex}{0ex}{0ex}

% Pengaturan persamaan
\newenvironment{conditions}
{\par\vspace{\abovedisplayskip}\noindent
	\tabularx{\columnwidth}{>{$}l<{$} @{${}={}$} >{\raggedright\arraybackslash}X}}
{\endtabularx\par\vspace{\belowdisplayskip}}

% Pengaturan format baris program
\usepackage{listings}
\definecolor{comment}{RGB}{0,128,0}
\definecolor{string}{RGB}{255,0,0}
\definecolor{keyword}{RGB}{0,0,255}
\lstdefinestyle{codestyle}{
	commentstyle=\color{comment},
	stringstyle=\color{string},
	keywordstyle=\color{keyword},
	basicstyle=\footnotesize\ttfamily,
	numbers=left,
	numberstyle=\tiny,
	numbersep=5pt,
	frame=lines,
	breaklines=true,
	prebreak=\raisebox{0ex}[0ex][0ex]{\ensuremath{\hookleftarrow}},
	showstringspaces=false,
	upquote=true,
	tabsize=2,
}
\lstset{style=codestyle}

% Isi keseluruhan dokumen
\begin{document}

  % Nomor halaman pembuka dimulai dari sini
  \pagenumbering{roman}

  % Sampul luar
  \AddToShipoutPictureBG*{
  \AtPageLowerLeft{
    % Ubah nilai berikut jika posisi horizontal background tidak sesuai
    \hspace{-3.5mm}

    % Ubah nilai berikut jika posisi vertikal background tidak sesuai
    \raisebox{0mm}{
      \includegraphics[width=\paperwidth,height=\paperheight]{sampul/sampul-luar.png}
    }
  }
}

% Menyembunyikan nomor halaman
\thispagestyle{empty}

% Pengaturan margin untuk menyesuaikan konten sampul
\newgeometry{top=70mm,left=25mm,right=20mm,bottom=25mm}

\begin{flushleft}

  % Pengaturan jenis dan warna teks yang digunakan
  \sffamily\color{white}

  % Ubah penomoran buku berikut sesuai dengan yang ditentukan oleh departemen
  \noindent\textbf{KERJA PRAKTIK – EC184601}
  \vspace{4ex}

  % Ubah kalimat berikut sesuai dengan nama perusahaan tempat kerja praktik
  \noindent{\large \textbf{PT. ANEKA TUNA INDONESIA}} \\
  % Ubah tanggal berikut sesuai dengan waktu berlangsungnya kerja praktik
  \textbf{(06 Juli 2020 s/d 11 September 2020)}
  \vspace{6ex}

  % Ubah kalimat berikut sesuai dengan judul topik kerja praktik
  \noindent{\large \textbf{PEMBUATAN \emph{PROGRESSIVE WEB APP} UNTUK ADMINISTRASI LOADING BARANG BERBASIS \emph{NODE.JS}}}
  \vspace{6ex}

  \begin{adjustwidth}{-0.2cm}{}
    \begin{tabular}{lcp{0.7\linewidth}}
      % Ubah kalimat-kalimat berikut sesuai dengan nama dan NRP mahasiswa pertama
      \textbf{Muhammad Alfi Maulana Fikri} & & \textbf{NRP 0721 17 4000 0009} \\
      % Ubah kalimat-kalimat berikut sesuai dengan nama dan NRP mahasiswa kedua
      \textbf{Fairuz Fadilah Soemarsono} & & \textbf{NRP 0721 17 4000 0033} \\
    \end{tabular}
  \end{adjustwidth}
  \vspace{4ex}

  \noindent\textbf{Dosen Pembimbing} \\
  % Ubah kalimat berikut sesuai dengan nama dosen pembimbing
  \textbf{Arief Kurniawan, S.T., M.T.}
  \vspace{12ex}

  % Ubah kalimat berikut sesuai dengan nama departemen
  \noindent\textbf{DEPARTEMEN TEKNIK KOMPUTER} \\
  % Ubah kalimat berikut sesuai dengan nama fakultas
  \textbf{Fakultas Teknologi Elektro dan Informatika Cerdas} \\
  % Ubah kalimat berikut sesuai dengan nama universitas
  \textbf{Institut Teknologi Sepuluh Nopember} \\
  % Ubah kalimat berikut sesuai dengan tempat dan tahun pembuatan buku
  \textbf{Surabaya 2020}

\end{flushleft}

\restoregeometry
  \cleardoublepage

  % Sampul dalam
  \AddToShipoutPictureBG*{
  \AtPageLowerLeft{
    % Ubah nilai berikut jika posisi horizontal background tidak sesuai
    \hspace{-3.5mm}

    % Ubah nilai berikut jika posisi vertikal background tidak sesuai
    \raisebox{0mm}{
      \includegraphics[width=\paperwidth,height=\paperheight]{sampul/sampul-dalam.png}
    }
  }
}

% Pengaturan margin untuk menyesuaikan konten sampul
\newgeometry{top=70mm,left=25mm,right=20mm,bottom=25mm}

\begin{flushleft}

  % Pengaturan jenis teks yang digunakan
  \sffamily

  % Ubah penomoran buku berikut sesuai dengan yang ditentukan oleh departemen
  \noindent\textbf{KERJA PRAKTIK – EC184601}
  \vspace{4ex}

  % Ubah kalimat berikut sesuai dengan nama perusahaan tempat kerja praktik
  \noindent{\large \textbf{PT. ANEKA TUNA INDONESIA}} \\
  % Ubah tanggal berikut sesuai dengan waktu berlangsungnya kerja praktik
  \textbf{(06 Juli 2020 s/d 11 September 2020)}
  \vspace{6ex}

  % Ubah kalimat berikut sesuai dengan judul topik kerja praktik
  \noindent{\large \textbf{PEMBUATAN \emph{PROGRESSIVE WEB APP} UNTUK ADMINISTRASI LOADING BARANG BERBASIS \emph{NODE.JS}}}
  \vspace{6ex}

  \begin{adjustwidth}{-0.2cm}{}
    \begin{tabular}{lcp{0.7\linewidth}}
      % Ubah kalimat-kalimat berikut sesuai dengan nama dan NRP mahasiswa pertama
      \textbf{Muhammad Alfi Maulana Fikri} & & \textbf{NRP 0721 17 4000 0009} \\
      % Ubah kalimat-kalimat berikut sesuai dengan nama dan NRP mahasiswa kedua
      \textbf{Fairuz Fadilah Soemarsono} & & \textbf{NRP 0721 17 4000 0033} \\
    \end{tabular}
  \end{adjustwidth}
  \vspace{4ex}

  \noindent
  \textbf{Dosen Pembimbing} \\
  % Ubah kalimat berikut sesuai dengan nama dosen pembimbing
  \textbf{Arief Kurniawan, S.T., M.T.}
  \vspace{12ex}

  % Ubah kalimat berikut sesuai dengan nama departemen
  \noindent\textbf{DEPARTEMEN TEKNIK KOMPUTER} \\
  % Ubah kalimat berikut sesuai dengan nama fakultas
  \textbf{Fakultas Teknologi Elektro dan Informatika Cerdas} \\
  % Ubah kalimat berikut sesuai dengan nama universitas
  \textbf{Institut Teknologi Sepuluh Nopember} \\
  % Ubah kalimat berikut sesuai dengan tempat dan tahun pembuatan buku
  \textbf{Surabaya 2020}

\end{flushleft}

\restoregeometry
  \cleardoublepage

  % Lembar pengesahaan untuk departemen
  \begin{center}
  {\Large \textbf{LEMBAR PENGESAHAN}}
  \vspace{4ex}

  \addcontentsline{toc}{chapter}{LEMBAR PENGESAHAN (DEPARTEMEN)}

  % Ubah kalimat berikut sesuai dengan judul topik kerja praktik
  {\large \textbf{PEMBUATAN \emph{PROGRESSIVE WEB APP} UNTUK ADMINISTRASI LOADING BARANG BERBASIS \emph{NODE.JS}}} \\
  di \\
  PT. Aneka Tuna Indonesia
  \vspace{4ex}

  % Ubah kalimat berikut sesuai dengan kalimat pengesahan yang ditentukan oleh departemen
  Laporan Kerja Praktik ini disusun untuk memenuhi persyaratan akademik di Departemen Teknik Komputer FTEIC - ITS
  \vspace{2ex}

  % Ubah kalimat-kalimat berikut sesuai dengan tempat dan tanggal pengesahan
  Tempat Pengesahan di: Surabaya \\
  Tanggal: 18 September 2020
  \vspace{8ex}

  Menyetujui, \\
  Dosen Pembimbing,
  \vspace{12ex}

  % Ubah kalimat-kalimat berikut sesuai dengan nama dan NIP dosen pembimbing
  \textbf{\underline{Arief Kurniawan, S.T., M.T.}} \\
  NIP. 19740907 200212 1 001
  \vspace{8ex}

  Mengetahui, \\
  % Ubah kalimat berikut sesuai dengan jabatan kepala departemen
  Kepala Departemen Teknik Komputer FTEIC - ITS,
  \vspace{12ex}

  % Ubah kalimat-kalimat berikut sesuai dengan nama dan NIP kepala departemen
  \textbf{\underline{Dr. Supeno Mardi Susiki Nugroho, S.T., M.T.}} \\
  NIP 19700313 199512 1 001

\end{center}
  \cleardoublepage

  % Lembar pengesahan untuk perusahaan
  \begin{center}
  {\Large \textbf{LEMBAR PENGESAHAN}}
  \vspace{4ex}

  \addcontentsline{toc}{chapter}{LEMBAR PENGESAHAN (PERUSAHAAN)}

  % Ubah kalimat berikut sesuai dengan judul topik kerja praktik
  {\large \textbf{PEMBUATAN \emph{PROGRESSIVE WEB APP} UNTUK ADMINISTRASI LOADING BARANG BERBASIS \emph{NODE.JS}}} \\
  di \\
  PT. Aneka Tuna Indonesia
  \vspace{4ex}

  % Ubah kalimat berikut sesuai dengan kalimat pengesahan yang ditentukan oleh departemen
  Laporan Kerja Praktik ini disusun untuk memenuhi persyaratan akademik di Departemen Teknik Komputer FTEIC - ITS
  \vspace{2ex}

  % Ubah kalimat-kalimat berikut sesuai dengan tempat dan tanggal pengesahan
  Tempat Pengesahan di: Surabaya \\
  Tanggal: 18 September 2020
  \vspace{8ex}

  Mengetahui, \\
  Pembimbing Perusahaan
  \vspace{12ex}

  % Ubah kalimat berikut sesuai dengan nama pembimbing perusahaan
  \textbf{\underline{Pak Jarot}}
  \vspace{8ex}

  Mengetahui, \\
  % Ubah kalimat berikut sesuai dengan jabatan kepala perusahaan
  Chief Executive Officer PT. Aneka Tuna Indonesia
  \vspace{12ex}

  % Ubah kalimat berikut sesuai dengan nama kepala perusahaan.
  \textbf{\underline{Pak Lie}}

\end{center}
  \cleardoublepage

  % Kata pengantar
  \begin{center}
  \Large\textbf{KATA PENGANTAR}
\end{center}
\vspace{2ex}

\addcontentsline{toc}{chapter}{KATA PENGANTAR}

% Pengaturan ukuran indentasi
\setlength{\parindent}{7ex}

% Ubah paragraf-paragraf berikut sesuai dengan yang ingin diisi pada kata pengantar

Puji syukur kami panjatkan kepada Allah SWT karena hanya dengan rahmat dan hidayah-Nya Penulis dapat melaksanakan kerja praktik dan menyelesaikan laporan kerja praktik di PT. Aneka Tuna Indonesia dengan judul “PEMBUATAN \emph{PROGRESSIVE WEB APP} UNTUK ADMINISTRASI LOADING BARANG BERBASIS \emph{NODE.JS}”.
Kerja praktik telah dilaksanakan pada tanggal 6 Juli 2020 s.d. 11 September 2020.
Penulisan laporan kerja praktik ini disusun sebagai syarat untuk memenuhi mata kuliah Kerja Praktik di Departemen Teknik Komputer FTEIC - ITS.
\vspace{0.5ex}

Dalam pelakasanaan maupun penulisan laporan kerja praktik ini, Penulis mengucapkan terima kasih atas bantuan, arahan, dan motivasi yang diberikan baik secara langsung ataupun tidak langsung.
Adapun pihak-pihak yang telah membantu dan membimbing kami dalam pelaksanaan kerja praktik yaitu:
\vspace{0.5ex}

\begin{enumerate}[nolistsep]

  \item Allah SWT. yang dengan rahmat-Nya kami dapat melaksanakan kerja praktik dengan lancar,
  \vspace{0.5ex}

  \item Bapak Arief Kurniawan, S.T., M.T. selaku Dosen Pembimbing di Departemen Teknik Komputer FTE-ITS, yang telah memberikan bimbingan kepada kami selama mengerjakan kerja praktik di perusahaan.
  \vspace{0.5ex}

  \item Pak Jarot yang telah banyak memberikan arahan kepada kami.
  \vspace{0.5ex}

  \item Semua anggota PT. Aneka Tuna Indonesia yang telah memberikan ilmu-ilmu baru kepada kami serta berkenan untuk kami wawancarai maupun kami ajak berdiskusi.
  \vspace{0.5ex}

  \item Bapak  Dr. Supeno Mardi Susiki Nugroho, S.T., M.T. selaku Kepala Departemen Teknik Komputer yang telah memberikan bimbingannya kepada kami.
  \vspace{0.5ex}

  \item Orang tua dan keluarga tercinta yang telah mencurahkan doa dan semangat yang tiada henti.
  \vspace{0.5ex}

  \item Teman-teman Fakultas Teknologi Elektro ITS angkatan 2017 yang selalu membagikan informasi kepada kami.
  \vspace{0.5ex}

  \item Serta semua pihak yang tidak bisa disebutkan satu-persatu yang turut membantu dan memperlancar jalannya kerja praktik ini.
  \vspace{0.5ex}

\end{enumerate}
\vspace{0.5ex}

Penulis menyadari bahwa masih banyak kekurangan dalam perancangan dan pembuatan laporan Kerja Praktik ini.
Besar harapan penulis untuk menerima saran dan kritik dari para pembaca.
Semoga buku laporan kerja praktik ini dapat memberikan manfaat bagi para pembaca, khususnya bagi penulis sendiri.
\vspace{2ex}

\begin{flushright}
  \begin{tabular}[b]{c}
    % Ubah kalimat berikut sesuai dengan tempat, bulan, dan tahun penulisan
    Surabaya, 18 September 2020
    \\
    \\
    \\
    \\
    Penulis
  \end{tabular}
\end{flushright}
  \cleardoublepage

  % Daftar isi
  \renewcommand*\contentsname{DAFTAR ISI}
  \addcontentsline{toc}{chapter}{\contentsname}
  \titlespacing*{\chapter}{0pt}{0ex}{0ex}
  \tableofcontents
  \cleardoublepage

  % Daftar gambar
  \renewcommand*\listfigurename{DAFTAR GAMBAR}
  \addcontentsline{toc}{chapter}{\listfigurename}
  \titlespacing*{\chapter}{0pt}{0ex}{0ex}
  \listoffigures
  \cleardoublepage

  % Daftar tabel
  \renewcommand*\listtablename{DAFTAR TABEL}
  \addcontentsline{toc}{chapter}{\listtablename}
  \titlespacing*{\chapter}{0pt}{0ex}{0ex}
  \listoftables
  \cleardoublepage

  % Nomor halaman isi dimulai dari sini
  \pagenumbering{arabic}

  % Bab 1 pendahuluan
	% Ubah kalimat sesuai dengan judul dari bab ini
\chapter{PENDAHULUAN}
\vspace{4ex}

% Pengaturan ukuran indentasi
\setlength{\parindent}{7ex}

% Ubah konten-konten berikut sesuai dengan yang ingin diisi pada bab ini

\section{Latar Belakang}
\vspace{1ex}

Sebagai bentuk realisasi kebijaksanaan pemerintah dalam peningkatan mutu pendidikan perguruan tinggi dan untuk mendukung program link and match antara perguruan tinggi dengan dunia industri, maka diperlukan suatu bentuk kerja sama antara pihak perguruan tinggi dengan praktisi industri.
\vspace{0.5ex}

Salah satu bentuk kerja sama yang nyata adalah dengan pelaksanaan Kerja praktik seperti yang tercantum dalam kurikulum di perguruan tinggi, dalam hal ini Institut Teknologi Sepuluh Nopember Surabaya, di lingkungan perusahaan yang menerapkan teknologi yang sesuai dengan bidang studi mahasiswa.
Mata kuliah Kerja Praktik diharapkan dapat mendorong mahasiswa untuk mengenal kondisi di lapangan kerja dan untuk melihat keselarasan antara ilmu pengetahuan yang diperoleh di bangku kuliah dengan aplikasi praktis di dunia kerja.
\vspace{0.5ex}

PT. Aneka Tuna Indonesia sebagai salah satu perusahaan yang bergerak di bidang produksi ikan kaleng di Indonesia, diharapkan dapat menjembatani upaya-upaya perguruan tinggi dalam meningkatkan mutu pendidikannya melalui Kerja praktik sehingga membantu meningkatkan kualitas lulusan perguruan tinggi yang berdaya saing tinggi di dunia industri.
\vspace{0.5ex}

Dalam kerja praktik yang dilakukan, pekerjaan yang dilakukan adalah proses pembuatan \emph{Progressive Web App} Berbasis \emph{Node.js} untuk administrasi loading barang di PT. Aneka Tuna Indonesia.

\section{Waktu dan Tempat Pelaksanaan}
\vspace{1ex}

Kerja praktik dilakukan secara Work From Home, dengan dua kali kunjungan di pabrik PT. Aneka Tuna Indonesia, yang beralamant di Jl. Randupitu - Gunung Gangsir No. 36, Asabri, Nogosari, Kec. Pandaan, Pasuruan, Jawa Timur 67156, dari tanggal 6 Juli 2020 sampai dengan 11 September 2020.
\vspace{0.5ex}

\section{Tujuan}
\vspace{1ex}

Secara umum, tujuan dari kerja praktik ini adalah:
\vspace{0.5ex}

\begin{enumerate}[nolistsep]

  \item Menciptakan hubungan yang sinergis, jelas dan terarah antara dunia industri dan perguruan tinggi, dimana output perguruan tinggi merupakan sumber daya manusia dalam dunia industri.
  \vspace{0.5ex}

  \item Meningkatkan kepedulian dan partisipasi oleh dunia industri dalam memberikan kontribusi pada sistem pendidikan nasional.
  \vspace{0.5ex}

  \item Membuka wawasan mahasiswa agar dapat mengetahui dan memahami aplikasi ilmunya di dunia industri.
  \vspace{0.5ex}

  \item Sebagai sarana pembelajaran sosialisasi dalam lingkungan dunia kerja.
  \vspace{0.5ex}

  \item Mahasiswa dapat memahami dan mengetahui sistem kerja di dunia industri sekaligus mampu mengadakan pendekatan masalah yang ada.
  \vspace{0.5ex}

  \item Menumbuhkan dan menciptakan pola berpikir konstruktif yang lebih berwawasan bagi mahasiswa serta meningkatkan kemampuan praktik di bidang pengembangan aplikasi berbasis web dan mengaplikasikan langsung ilmu yang telah dipelajari di Departemen Teknik Komputer FTEIC - ITS.
  \vspace{0.5ex}

  \item Meningkatkan kedisiplinan, kemandirian, dan kepekaan mahasiswa melalui budaya kerja di dalam perusahaan.
  \vspace{0.5ex}

\end{enumerate}
\vspace{0.5ex}

Serta secara khusus, tujuan dari kerja praktik ini adalah:
\vspace{0.5ex}

\begin{enumerate}[nolistsep]

  \item Untuk memenuhi beban satuan kredit semester (SKS) yang harus ditempuh sebagai persyaratan akademis di Departemen Teknik Komputer FTEIC - ITS.
  \vspace{0.5ex}

  \item Mengembangkan pengetahuan, sikap, keterampilan dan kemampuan profesi melalui penerapan ilmu, latihan kerja dan pengamatan teknik yang diterapkan di PT. Aneka Tuna Indonesia.
  \vspace{0.5ex}

  \item Memperdalam pengetahuan mahasiswa dengan mengenal dan mempelajari secara langsung penerapan ilmu teknik komputer.
  \vspace{0.5ex}

  \item Mengembangkan hubungan baik antara pihak perguruan tinggi dengan PT. Aneka Tuna Indonesia.
  \vspace{0.5ex}

  \item Melakukan analisis dan memberikan rekomendasi dalam bentuk laporan kerja praktik kepada PT. Aneka Tuna Indonesia mengenai permasalahan yang dihadapi perusahaan.

\end{enumerate}
\vspace{0.5ex}

\section{Batasan Masalah}
\vspace{1ex}

Dalam penulisan laporan ini akan dibahas tentang pembuatan \emph{Progressive Web App} berbasis \emph{Node.js} untuk administrasi loading barang PT. Aneka Tuna Indonesia.
\vspace{0.5ex}

\section{Metodologi Pengumpulan Data}
\vspace{1ex}

Metodologi yang digunakan pada penyusunan laporan kerja ini adalah:
\vspace{0.5ex}

\begin{enumerate}[nolistsep]

  \item \textbf{Metode Eksperimen}
  \vspace{0.5ex}

  Penulis memperoleh data melalui percobaan langsung pada objek sehingga dapat mengamati pengaruh setiap komponen objek dan hubungan mereka disertai pencatatan tentang pengertian dan fungsi objek dengan singkat dan jelas.
  \vspace{0.5ex}

  \item \textbf{Studi Literatur}
  \vspace{0.5ex}

  Penulis mencatat atau memanfaatkan referensi berupa katalog, arsip-arsip, dan buku-buku. Referensi diperoleh dari perpustakaan dan dokumen perusahaan.
  \vspace{0.5ex}

  \item \textbf{Metode Diskusi}
  \vspace{0.5ex}

  Penulis mengumpulkan data melalui diskusi atau menanyakan secara langsung kepada pembimbing dan pegawai.
  Tujuannya untuk mendapatkan data-data secara langsung dan jelas.
  \vspace{0.5ex}

\end{enumerate}
\vspace{0.5ex}

\section{Sistematika Penulisan}
\vspace{1ex}

Dalam penulisan laporan kerja praktik ini, penulis membagi laporan dalam beberapa bab yang disusun dengan sistematika sebagai berikut:
\vspace{0.5ex}

\begin{enumerate}[nolistsep]

  \item \textbf{Bab I Pendahuluan}
  \vspace{0.5ex}

  Bab ini memaparkan mengenai garis besar kerja praktik yang meliputi latar belakang, waktu dan tempat pelaksanaan, tujuan kerja praktik, batasan masalah, metodologi pengumpulan data, serta sistematika penulisan laporan kerja praktik.
  \vspace{0.5ex}

  \item \textbf{Bab II Profil Perusahaan}
  \vspace{0.5ex}

  Bab ini berisi penjelasan mengenai profil PT. Aneka Tuna Indonesia yang meliputi sejarah, visi dan misi, dan struktur organisasi yang ada di perusahaan tersebut.
  \vspace{0.5ex}

  \item \textbf{Bab III Tinjauan Pustaka}
  \vspace{0.5ex}

  Bab ini berisi penjelasan tentang istilah-istilah atau teori-teori yang digunakan dalam pembuatan kerja praktik dan pustaka yang dipakai.
  \vspace{0.5ex}

  \item \textbf{Bab IV Desain dan Implementasi}
  \vspace{0.5ex}

  Bab ini berisi pemaparan mengenai kebutuhan untuk perancangan beserta implementasi dari sistem yang akan dibangun dan dikembangkan.
  \vspace{0.5ex}

  \item \textbf{Bab V Pengujian dan Evaluasi}
  \vspace{0.5ex}

  Bab ini berisi penjelasan tentang hasil pengujian sistem dan evaluasi yang dilakukan terhadap kinerja sistem secara menyeluruh.
  \vspace{0.5ex}

  \item \textbf{Bab VI Kesimpulan dan Saran}
  \vspace{0.5ex}

  Bab ini berisi kesimpulan dan saran dari proses selama pengerjaan kerja praktik di PT. Aneka Tuna Indonesia.
  \vspace{0.5ex}

\end{enumerate}
\vspace{0.5ex}

  \cleardoublepage

  % Bab 2 profil perusahaan
	% Ubah kalimat sesuai dengan judul dari bab ini
\chapter{PROFIL PERUSAHAAN}
\vspace{4ex}

% Pengaturan ukuran indentasi
\setlength{\parindent}{7ex}

% Ubah konten-konten berikut sesuai dengan yang ingin diisi pada bab ini

\section{Sejarah PT. Aneka Tuna Indonesia}
\vspace{1ex}

PT. Aneka Tuna Indonesia berdiri pada \lipsum[1]
\vspace{0.5ex}

\lipsum[2]
\vspace{0.5ex}

\section{Visi dan Misi}
\vspace{1ex}

PT. Aneka Tuna Indonesia memiliki visi dan misi sebagai berikut:
\vspace{0.5ex}

\begin{enumerate}[nolistsep]

  \item \textbf{Visi PT. Aneka Tuna Indonesia}
  \vspace{0.5ex}

  Menjadi \lipsum[1][1-3]
  \vspace{0.5ex}

  \item \textbf{Misi PT. Aneka Tuna Indonesia}
  \vspace{0.5ex}

  \begin{enumerate}[nolistsep]

    \item Membuat \lipsum[1][1-2]
    \vspace{0.5ex}

    \item \lipsum[1][3-4]
    \vspace{0.5ex}

  \end{enumerate}
  \vspace{0.5ex}

\end{enumerate}
\vspace{0.5ex}

\section{Struktur Organisasi}
\vspace{1ex}

Struktur Organisasi dari \lipsum[1]
\vspace{0.5ex}

% Contoh input gambar dengan format *.png
\begin{figure} [ht] \centering
  % Nama dari file gambar yang diinputkan
  \includegraphics[scale=0.45]{gambar/struktur-organisasi.png}
  % Keterangan gambar yang diinputkan
  \caption{Struktur Organisasi PT. Aneka Tuna Indonesia}
  % Label referensi dari gambar yang diinputkan
	\label{fig:strukturOrganisasi}
\end{figure}

% Contoh penggunaan referensi dari gambar yang diinputkan
Seperti yang bisa dilihat pada \ref{fig:strukturOrganisasi}, \lipsum[1]
\vspace{0.5ex}

  \cleardoublepage

  % Bab 3 tunjauan pustaka
	% Ubah kalimat sesuai dengan judul dari bab ini
\chapter{TINJAUAN PUSTAKA}
\vspace{4ex}

% Pengaturan ukuran indentasi
\setlength{\parindent}{7ex}

% Ubah konten-konten berikut sesuai dengan yang ingin diisi pada bab ini

\section{\emph{Node.js}}
\vspace{1ex}
Node.js adalah open source, cross-platform, back end, JavaScript runtime environment yang mengeksekusi kode 
JavaScript di luar web browser. 
Node.js memungkinkan pengembang menggunakan JavaScript untuk menulis alat baris 
perintah dan skrip untuk menjalankan server-side scripting untuk menghasilkan konten halaman web dinamis sebelum
halaman dikirim ke browser Web pengguna. 
Node.js menyatukan pengembangan aplikasi web di satu bahasa pemrograman.
\vspace{0.5ex}

\section{\emph{Document-oriented Database}}
\vspace{1ex}
Document-oriented database adalah program komputer yang dirancang untuk menyimpan, mengambil, dan mengelola 
informasi berorientasi dokumen.
Document-oriented database adalah salah satu kategori utama database NoSQL. 
Konsep sentral dari Document-oriented database adalah pengertian tentang dokumen. Secara umum, 
mereka semua menganggap dokumen mengenkapsulasi dan menyandikan data (atau informasi) dalam beberapa format 
standar atau pengkodean. Pengkodean yang digunakan termasuk XML, YAML, JSON, serta bentuk biner seperti BSON.
\vspace{0.5ex}

\subsection{\emph{MongoDB}}
\vspace{1ex}
MongoDB adalah program cross-platform untuk document-oriented database dan diklasifikasikan sebagai program database 
NoSQL. MongoDB menggunakan dokumen dengan skema opsional yang fleksibel seperti JSON, yang berarti bidang dapat 
bervariasi dari dokumen ke dokumen dan struktur data dapat diubah seiring waktu. MongoDB adalah basis data yang 
terdistribusi pada intinya, sehingga ketersediaan, penskalaan, dan distribusi dapat dengan mudah 
dibangun dan digunakan.
\vspace{0.5ex}

\subsection{\emph{Mongoose}}
\vspace{1ex}

\lipsum[4]
\vspace{0.5ex}

\section{\emph{Representational State Transfer API} (\emph{REST API})}
\vspace{1ex}
Representational state transfer (REST) adalah gaya arsitektur perangkat lunak yang mendefinisikan sekumpulan constrain
yang akan digunakan untuk membuat layanan Web. Layanan web yang sesuai dengan gaya arsitektur REST disebut 
layanan Web RESTful yang menyediakan interoperabilitas antara sistem komputer di internet. Layanan Web RESTful 
memungkinkan sistem yang meminta untuk mengakses dan memanipulasi representasi tekstual dari Web resources
dengan menggunakan serangkaian operasi tanpa pernyataan yang seragam dan telah ditentukan sebelumnya.

\lipsum[5]
\vspace{0.5ex}

\subsection{\emph{Express}}
\vspace{1ex}
Express.js, atau Express, adalah kerangka aplikasi web back end untuk Node.js, dirilis sebagai free open-source 
software di bawah Lisensi MIT. Express.js dirancang untuk membangun aplikasi web dan API dan merupakan 
framework server standar untuk Node.js. Express js. dideskripsikannya sebagai server yang terinspirasi dari Sinatra, 
yang berarti bahwa server ini relatif minimalis dengan banyak fitur yang tersedia sebagai plugin. Express adalah 
komponen back-end dari MEAN stack, bersama dengan perangkat lunak database MongoDB dan framework front-end AngularJS.
\vspace{0.5ex}

\subsection{\emph{Axios}}
\vspace{1ex}

\lipsum[7]
\vspace{0.5ex}

\section{\emph{Progressive Web App} (\emph{PWA})}
\vspace{1ex}
Progressive Web App (PWA) adalah jenis perangkat lunak aplikasi yang dikirimkan melalui web, dibuat menggunakan 
teknologi web umum seperti HTML, CSS, dan JavaScript untuk bekerja pada platform apa pun yang menggunakan browser 
yang sesuai standar, termasuk desktop dan perangkat seluler.
Fitur PWA memungkinkan untuk menutup celah ke aplikasi asli dan menciptakan pengalaman pengguna yang serupa seperti 
bekerja secara offline, performa yang cepat, akses ke dalam sensor ponsel, dukungan untuk push notification, dan 
ikon di layar beranda ponsel.
\vspace{0.5ex}

\subsection{\emph{Vue.js}}
\vspace{1ex}
Vue.js adalah open-source framework front-end JavaScript untuk membangun antarmuka 
pengguna dan aplikasi pada satu halaman.
Vue.js menampilkan arsitektur yang dapat disesuaikan secara bertahap yang berfokus pada rendering deklaratif dan 
komposisi komponen dengan inti Library yang difokuskan pada lapisan tampilan saja. Fitur-fitur canggih yang 
diperlukan untuk aplikasi kompleks seperti perutean, manajemen status, dan perkakas build ditawarkan melalui 
Library dan paket pendukung yang dikelola secara resmi.
Vue.js memungkinkan kita untuk memperluas HTML dengan atribut HTML yang disebut directives. Directives menawarkan 
fungsionalitas ke aplikasi HTML, dan datang sebagai bawaan atau yang ditentukan pengguna.
\vspace{0.5ex}

\subsection{\emph{Vuetify}}
\vspace{1ex}
Vuetify adalah Library antarmuka Vue dengan Komponen Material untuk memperindah tampilan. 
Tujuan Vuetify adalah menyediakan semua yang dibutuhkan pengguna untuk membangun aplikasi web yang indah dan 
menarik menggunakan spesifikasi Desain Material dan dengan siklus pembaruan yang konsisten, 
Dukungan Jangka Panjang (LTS), keterlibatan komunitas yang responsif, ekosistem sumber daya yang luas, 
dan dedikasi pada komponen berkualitas.
\vspace{0.5ex}
  \cleardoublepage

  % Bab 4 desain dan implementasi
	% Ubah kalimat sesuai dengan judul dari bab ini
\chapter{DESAIN DAN IMPLEMENTASI}
\vspace{4ex}

% Pengaturan ukuran indentasi
\setlength{\parindent}{7ex}

% Ubah konten-konten berikut sesuai dengan yang ingin diisi pada bab ini

\section{Deskripsi Sistem}
\vspace{1ex}

\lipsum[1]
\vspace{0.5ex}

\section{Spesifikasi Kasus Penggunaan}

\lipsum[2]
\vspace{0.5ex}

\section{Implementasi Database}
\vspace{1ex}

\lipsum[3]
\vspace{0.5ex}

\section{Implementasi \emph{REST API}}
\vspace{1ex}

\lipsum[4]
\vspace{0.5ex}

\section{Implementasi Aplikasi}
\vspace{1ex}

Aplikasi diimplementasikan dengan \lipsum[2]
\vspace{0.5ex}

% Digunakan untuk page break
\newpage

% Contoh pembuatan code snippet
\begin{lstlisting}[
  language=C++,
  label={lst:helloWorld},
  caption={Hello World}
]
#include <iostream>

int main() {
    std::cout << "Hello World!";
    return 0;
}
\end{lstlisting}
\vspace{0.5ex}

% Contoh penggunaan referensi dari code snippet yang diinputkan
Seperti contoh pada baris program \ref{lst:helloWorld} dan \ref{lst:bilanganPrima}, \lipsum[3]
\vspace{0.5ex}

% Contoh input code snippet
\lstinputlisting[
  % Bahasa yang digunakan oleh code snippet
  language=Python,
  % Label referensi dari code snippet yang diinputkan
  label={lst:bilanganPrima},
  % Keterangan dari code snippet yang diinputkan
  caption={Perhitungan Bilangan Prima}
% Nama dari file code snippet yang diinputkan
]{program/prime-number.py}
\vspace{0.5ex}
  \cleardoublepage

  % Bab 5 pengujian dan evaluasi
	% Ubah kalimat sesuai dengan judul dari bab ini
\chapter{PENGUJIAN DAN EVALUASI}
\vspace{4ex}

% Pengaturan ukuran indentasi
\setlength{\parindent}{7ex}

% Ubah konten-konten berikut sesuai dengan yang ingin diisi pada bab ini

\section{Lingkungan Pengujian}
\vspace{1ex}

\lipsum[1]
\vspace{0.5ex}

\section{Skenario Pengujian}
\vspace{1ex}

Pengujian dilakukan dengan \lipsum[1]
\vspace{0.5ex}

\section{Evaluasi Pengujian}
\vspace{1ex}

Dari pengujian yang \lipsum[2]
\vspace{0.5ex}

% Contoh penggunaan referensi dari tabel yang dibuat
Sesuai dengan hasil pada \ref{tb:energiKecepatan} didapatkan \lipsum[3]

% Contoh input konten dari file terpisah
\input{tabel/energi-kecepatan.tex}
\vspace{1ex}

\lipsum[4]
\vspace{0.5ex}
  \cleardoublepage

  % Bab 6 kesimpulan dan saran
	% Ubah kalimat sesuai dengan judul dari bab ini
\chapter{KESIMPULAN DAN SARAN}
\vspace{4ex}

% Pengaturan ukuran indentasi
\setlength{\parindent}{7ex}

% Ubah konten-konten berikut sesuai dengan yang ingin diisi pada bab ini

\section{Kesimpulan}
\vspace{1ex}

Kesimpulan yang kami peroleh dari hasil kerja praktik ini, antara lain:
\vspace{0.5ex}

\begin{enumerate}[nolistsep]

  \item Pembuatan \lipsum[1][1-2]
  \vspace{0.5ex}

  \item \lipsum[1][3-4]
  \vspace{0.5ex}

  \item \lipsum[1][5-6]
  \vspace{0.5ex}

\end{enumerate}
\vspace{0.5ex}

\section{Saran}
\vspace{1ex}

Penulis menyadari pentingnya keberadaan fitur baru yang telah dibuat ini, namun penulis menemukan beberapa hal yang kami rasa perlu untuk diperbaiki dan ditingkatkan, antara lain:
\vspace{0.5ex}

\begin{enumerate}[nolistsep]

  \item Sebaiknya \lipsum[1][1-2]
  \vspace{0.5ex}

  \item \lipsum[1][3-4]
  \vspace{0.5ex}

  \item \lipsum[1][5-6]
  \vspace{0.5ex}

\end{enumerate}
\vspace{0.5ex}
  \cleardoublepage

  % Daftar pustaka
  \renewcommand\bibname{DAFTAR PUSTAKA}
  \addcontentsline{toc}{chapter}{\bibname}
  \titlespacing*{\chapter}{0pt}{0ex}{5ex}
  \appendix
  \bibliographystyle{apalike}
  \bibliography{pustaka/pustaka.bib}
  \cleardoublepage

  % Biografi penulis
	%\input{lainnya/biografi-penulis.tex}
  %\cleardoublepage

\end{document}