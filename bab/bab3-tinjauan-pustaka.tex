% Ubah kalimat sesuai dengan judul dari bab ini
\chapter{TINJAUAN PUSTAKA}
\vspace{4ex}

% Pengaturan ukuran indentasi
\setlength{\parindent}{7ex}

% Ubah konten-konten berikut sesuai dengan yang ingin diisi pada bab ini

\section{\emph{Node.js}}
\vspace{1ex}

\lipsum[1]
\vspace{0.5ex}

\section{\emph{Document-oriented Database}}
\vspace{1ex}

\lipsum[2]
\vspace{0.5ex}

\subsection{\emph{MongoDB}}
\vspace{1ex}

\lipsum[3]
\vspace{0.5ex}

\subsection{\emph{Mongoose}}
\vspace{1ex}

\lipsum[4]
\vspace{0.5ex}

\section{\emph{Representational State Transfer API} (\emph{REST API})}
\vspace{1ex}

\lipsum[5]
\vspace{0.5ex}

\subsection{\emph{Express}}
\vspace{1ex}

\lipsum[6]
\vspace{0.5ex}

\subsection{\emph{Axios}}
\vspace{1ex}

\lipsum[7]
\vspace{0.5ex}

\section{\emph{Progressive Web App} (\emph{PWA})}
\vspace{1ex}

\lipsum[8]
\vspace{0.5ex}

\subsection{\emph{Vue.js}}
\vspace{1ex}

\lipsum[9]
\vspace{0.5ex}

\subsection{\emph{Vuetify}}
\vspace{1ex}

\lipsum[10]
\vspace{0.5ex}